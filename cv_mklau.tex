\documentclass[]{article}
\usepackage{multicol}
\usepackage[T1]{fontenc}
\usepackage{lmodern}
\usepackage{amssymb,amsmath}
\usepackage{ifxetex,ifluatex}
\usepackage{fixltx2e} % provides \textsubscript
% use upquote if available, for straight quotes in verbatim environments
\IfFileExists{upquote.sty}{\usepackage{upquote}}{}
\ifnum 0\ifxetex 1\fi\ifluatex 1\fi=0 % if pdftex
  \usepackage[utf8]{inputenc}
\else % if luatex or xelatex
  \ifxetex
    \usepackage{mathspec}
    \usepackage{xltxtra,xunicode}
  \else
    \usepackage{fontspec}
  \fi
  \defaultfontfeatures{Mapping=tex-text,Scale=MatchLowercase}
  \newcommand{\euro}{€}
\fi
% use microtype if available
\IfFileExists{microtype.sty}{\usepackage{microtype}}{}
\ifxetex
  \usepackage[setpagesize=false, % page size defined by xetex
              unicode=false, % unicode breaks when used with xetex
              xetex]{hyperref}
\else
  \usepackage[unicode=true]{hyperref}
\fi
\hypersetup{breaklinks=true,
            bookmarks=true,
            pdfauthor={},
            pdftitle={},
            colorlinks=true,
            citecolor=blue,
            urlcolor=blue,
            linkcolor=magenta,
            pdfborder={0 0 0}}
\urlstyle{same}  % don't use monospace font for urls
\setlength{\parindent}{0pt}
\setlength{\parskip}{6pt plus 2pt minus 1pt}
\setlength{\emergencystretch}{3em}  % prevent overfull lines
\setcounter{secnumdepth}{0}

\author{}
\date{}

\begin{document}

\textbf{\Large Matthew K. Lau} \\

\hline
\begin{multicols}{2}
\textit{Postdoctoral Fellow} \\
Harvard Forest \\
Harvard University \\
324 N. Main St. \\
Petersham, MA 01366 \\
Office: (978) 756-6165 \\
Email: matthewklau@fas.harvard.edu \\
Website: \url{https://github.com/MKLau} \\
\end{multicols}

\section{Education}\label{education}

\begin{itemize}
\item
  Ph.D.~Biology, \href{http://www.nau.edu}{Northern Arizona University},
  \href{http://www.mpcer.nau.edu/igert/}{IGERT Fellow}, 2009--2014

  \begin{itemize}
  \item
    \emph{Title}: The evolution of ecological networks.
  \item
    \emph{Advisor}:
    \href{http://www6.nau.edu/biology/People/Faculty/Whitham/Whitham.htm}{Dr.~Thomas
    G. Whitham}
  \end{itemize}
\item
  M.S. Biology, \href{http://www.nau.edu}{Northern Arizona University},
  2008

  \begin{itemize}
  \item
    \emph{Thesis}: Host species and site contribute to variation in
    foliar endophyte abundance, diversity and community composition.
  \item
    \emph{Advisor}:
    \href{http://www.nau.edu/~envsci/johnsonlab/index.htm}{Dr.~Nancy C.
    Johnson}
  \end{itemize}
\item
  B.S. Biology, \href{http://www.humboldt.edu/~biosci/}{Humboldt State
  University}, 2004

  \begin{itemize}
  \item
    \emph{Emphasis}: Fungal Ecology
  \item
    \emph{Advisors}: Dr.~Nathan J. Sanders and Dr.~Terry W. Henkel
  \item
    \emph{Advanced Study}:
    \href{http://harvardforest.fas.harvard.edu/education/reu/reu.html}{Harvard
    Forest REU}, Summer 2003
  \item
    ~~~ \emph{Mentor}:
    \href{http://harvardforest.fas.harvard.edu/profiles/ellison.html}{Dr.~Aaron
    M. Ellison}
  \end{itemize}
\end{itemize}

\section{Awards and Fellowships}\label{awards-and-fellowships}

\begin{itemize}
\item
  ARCS Foundation Scholarship (Phoenix Chapter), 2013--2014
\item
  Chateaubriand Fellowship (French Embassy), 2011--2012
\item
  NSF IGERT Fellowship (Northern Arizona University), 2008--2010
\item
  ARCS Foundation Scholarship (Phoenix Chapter), 2007--2008
\item
  ARCS Foundation Scholarship (Phoenix Chapter), 2006--2007
\item
  Northern Arizona University Minority Student Development Fellowship,
  2005--2008
\end{itemize}

\section{Computer Software and Language
Proficiencies}\label{computer-software-and-language-proficiencies}

Computer: R, Python, Matlab, LaTeX, HTML, Bash, Emacs, git, MacOS,
Linux/Unix and Windows\\\\Human: English (native speaker), French (not
fluent), Mandarin (not fluent) and Spanish (not fluent)

\section{Research Experience}\label{research-experience}

\begin{itemize}
\item
  Visiting Researcher,
  \href{https://www4.bordeaux-aquitaine.inra.fr/biogeco/}{Community
  Genetics Laboratory}
  (\href{http://www4.bordeaux-aquitaine.inra.fr/biogeco_eng/People/Former-members/Michalet-Richard}{Dr.
  Richard Michalet}), Fall 2011
\item
  Visiting Researcher, \href{http://people.uncw.edu/borretts/}{Systems
  Ecology and Ecoinformatics Lab}
  (\href{http://people.uncw.edu/borretts/people.html}{Dr.~Stuart
  Borrett}), Summer 2011
\item
  Research Assistant, \href{http://www.poplar.nau.edu/}{Cottonwood
  Ecology Group}
  (\href{http://www.poplar.nau.edu/people.php?mode=showus\&user=tgw}{Dr.~Thomas
  G. Whitham}), 2010--2011
\item
  Research Assistant,
  \href{http://www.nau.edu/~envsci/johnsonlab/index.htm}{The Soil
  Ecology Lab}
  (\href{http://www.nau.edu/~envsci/johnsonlab/NCJ.htm}{Dr.~Nancy C.
  Johnson}), 2005--2008
\item
  Research Assistant, Humboldt State University Ant Ecology Lab
  \href{http://web.utk.edu/~nsanders/nate.html}{(Dr. Nathan J.
  Sanders)}, 2003
\end{itemize}

\section{Teaching Experience}\label{teaching-experience}

\begin{itemize}
\item
  Seminar on Ethics in Computional Ecology (Harvard Forest, Harvard
  University), Summer 2014
\item
  BIO326L Ecology Lab (Northern Arizona University), Fall 2012--Spring
  2013
\item
  \href{http://people.uncw.edu/borretts/courses/RworkshopUNCW.pdf}{Introduction
  to Programming in R (University of North Carolina Wilmington)}, Summer
  2011
\item
  BIO181 Introductory Biology: The Unity of Life (Northern Arizona
  Unversity), Fall 2010
\item
  BIO680
  \href{http://www.mpcer.nau.edu/igert/eco_analysis_r.html}{Introduction
  to Ecological Analyses in R} (Northern Arizona Unversity), Fall 2008
  and Fall 2009
\end{itemize}

\section{Contributed Software}\label{contributed-software}

{Matthew K. Lau, Stuart R. Borrett and David E. Hines (2014) enaR: Tools
for Ecological Network Analysis. R package version 2.7.}

{Matthew K. Lau and Raj Whitlock (2009) DiversitySampler: Functions for
re-sampling a community matrix to compute diversity indices at different
sampling levels.. R package version 2.0.}

{Matthew K. Lau (2009) DTK: Dunnett-Tukey-Kramer Pairwise Multiple
Comparison Test Adjusted for Unequal Variances and Unequal Sample Sizes.
R package version 3.0.}

\section{Publications}\label{publications}

\begin{itemize}
\item
  Lau~MK (2014) BOOK REVIEW: Grounding ecological networks. Ecology.
  95:2681--2682.
\item
  Flores-Rentería~L, Lau~MK, Lamit~LJ, Gehring CA (2014) An elusive
  ectomycorrhizal fungus reveals itself: A new species of Geopora
  (Pyronemataceae) associated with Pinus edulis. Mycologia. DOI
  10.3852/13-263.
\item
  Lamit~LJ, Lau~MK, Sthultz~CM, Wooley~SC, Whitham~TG, \& Gehring~CA

  \begin{enumerate}
  \def\labelenumi{(\arabic{enumi})}
  \setcounter{enumi}{2013}
  \itemsep1pt\parskip0pt\parsep0pt
  \item
    Tree genotype and genetically based growth traits structure twig
    endophyte communities. American Journal of Botany. DOI
    10.3732/ajb.1400034.
  \end{enumerate}
\item
  Ikeda~DH, Bothwell~HM, Lau~MK, O'Neill~G, Grady~K, Ferrier~SM,
  Allan~G, Shuster~SM \& Whitham TG (2013) A genetics-based Universal
  Community Transfer Function for predicting the impacts of climate
  change on future communities. Functional Ecology 28:65--74.
\item
  Lau~MK, Arnold~EA \& Johnson~NC (2013)Factors influencing communities
  of foliar fungal endophytes in riparian woody plants. Fungal Ecology
  6: 365--378.
\item
  Álvarez-Sánchez~FJ, Johnson~NC, Antoninka~AJ, Chaudhary~VB, Lau~MK,
  Owen~SM, Sánchez-Gallen~I, Guadarrama~P, \& Castillo S (2012)
  Large-scale diversity patterns in spore communities of arbuscular
  mycorrhizal fungi. In M. Pagano, editor, \emph{Mycorrhiza: Occurrence
  in Natural and Restored Environments}, Nova Science Publishers, New
  York (USA).
\item
  Bowker~MA, Muñoz~A, Martinez~T \& Lau~MK 2012 Rare drought-induced
  mortality of juniper is enhanced by edaphic stressors and influenced
  by stand density. Journal of Arid Environments 76:9--16.
\item
  Lau~MK, Whitham~TG, Lamit~LJ \& Johnson~NC (2010) Ecological \&
  Evolutionary Interaction Network Exploration: Addressing the
  Complexity of Biological Interactions in Natural Systems with
  Community Genetics and Statistics. JIFS 7:17--25
\item
  Price~LB, Johnson~KE, Aziz~M, Lau~MK, Bowers~J, Ravel~J, Keim~PS,
  Serwadda~D, Wawer~MJ \& Gray~RH (2010) The effects of circumcision on
  the penis microbiome. PLoS One 5(1):e8422.
\item
  Chaudhary~VB, Lau~MK \& Johnson~NC (2008) Macroecology of microbes --
  biogeography of the Glomeromycota. In V. Ajit, editor,
  \emph{Mycorrhiza} (3rd Edition), Springer-Verlag, Germany.
\item
  Ellison~AM, Chen~J, Diaz~D, Kammerer-Burnham~C \& Lau~M (2005) Changes
  in ant community structure and composition associated with hemlock
  decline in New England. Pages 280-289 in B. Onkenand and R. Reardon,
  editors. \emph{Proceedings of the 3rd Symposium on Hemlock Woolly
  Adelgid in the Eastern United States}. U.S. Department of Agriculture
  -- U.S. Forest Service -- Forest Health Technology Enterprise Team,
  Morgantown, West Virginia.

  \subsection{Manuscripts In Progress}\label{manuscripts-in-progress}

  \begin{itemize}
  \item
    Lau~MK, Borrett~SR, Keith~AR, Shuster~SM \& Whitham~TG (In Prep)
    Genotypic variation in foundation species generate ecological
    network structure.
  \item
    Butterfield~BJ, Lau~MK, Shutters~S, (In Review Ecology Letters)
    Merging positive and negative interaction networks provides new
    insights into community assembly rules.
  \item
    Floate~KD, Godbout~J, Lau~MK, Whitham~TG, Isabel~N (Submitted)
    Plant-herbivore interactions in a trispecific hybrid swarm of
    cottonwoods: Genetic similarity and the hybrid bridge hypothesis.
  \item
    Smith~DS, Lamit~LJ, Lau~MK, Gehring~CA \& Whitham~TG (In Review
    Ecological Entomology) Change of plant traits by introduced elk
    negatively affects associated arthropod communities and network
    structure.
  \item
    Borrett~SR \& Lau~MK (Submitted) An open-source package for
    ecological network analysis in R.
  \item
    Stone~AC, Gehring~CA, Lau~MK, Cobb~NS \& Whitham TG (In Prep) Plant
    mediated indirect effects on communities reduce diversity andPlant
    mediated indirect genetic effects of scale herbivory alter arthropod
    community networks on a foundation tree.
  \end{itemize}
\end{itemize}

\section{Presentations}\label{presentations}

\begin{itemize}
\item
  Lau~MK, Borrett~SR \emph{enaR: Free, open-source tools for ecological
  network analysis.} Ecological Society of America Meeting (ESA),
  Minneapolis, MN, August 2013
\item
  Lau~MK, Lamit~LJ, Gehring~CA, and Whitham TG \emph{Cottonwood genetics
  influence lichen interaction network structure.} Université Bordeaux
  1, Talence, France, December 2011
\item
  Whitham~TG, Lau~MK, Lamit~LJ, Smith~DS, Busby~PE, Schweitzer~JA,
  Gehring~CA, Allan~GJ, Shuster~SM and Newcombe~G * A Community Genetics
  Approach for Understanding Microbial Community Structure and Feedbacks
  on a Foundation Tree Species.* Ecological Society of America Meeting
  (ESA), Pittsburgh, PA, August 2010
\item
  Lau~MK, Keith~AR and Whitham~TG \emph{Network structure is linked to
  the community stability of canopy arthropods associated with Populus
  angustifolia.} Ecological Society of America Meeting (ESA),
  Pittsburgh, PA, August 2010
\item
  Lau~MK, Johnson~NC, Whitham~TG, Hagenauer~LE, Lamit~LJ and Lonsdorf~EV
  \emph{A Community Genetics Approach for Understanding Complex
  Biological Interactions.} 7th International Symposium on Integrated
  Field Science, Tohoku University, Sendai, Japan, October 2009
\item
  Lau~MK, Hagenauer~LE and Whitham~TG \emph{Assemblage-structuring force
  of species interactions varies spatially and temporally: Co-occurrence
  analysis of canopy arthropod distributions.} Ecological Society of
  America Meeting (ESA), Albuquerque, NM, August 2009
\item
  Lau~MK, Johnson~NC \emph{Fungal foliar endophyte communities exhibit
  host species fidelity in woody plants of Arizona riparian forests.}
  Ecological Society of America Meeting (ESA), Milwaukee, WI, August
  2008
\item
  Lau~MK \emph{Unusual absence of asymptomatic fungal leaf endophytes of
  Populus fremontii: a potential phytochemical mechanism}. (poster)
  Ecological Society of America Meeting (ESA), San Jose, CA, August 2007
\item
  Whitewater~L, Lau~MK, Johnson~NC \emph{Investigating the potential for
  local adaptation of the arbuscular mycorrhizal fungus} . (poster) REU
  Summer Research Symposium, Northern Arizona University, Aug 2007.
\item
  Lau~MK, Johnson~NC \emph{Do AMF cultivate their favorite bacteria? A
  hypothesis for a potential mechanism of AMF adaptation}. (poster) 5th
  International Conference on Mycorrhiza (ICOM5), Granada, Spain, July
  2006

\end{itemize}

\section{Professional Activities}\label{professional-activities}

\subsection{Article Reviewer}\label{article-reviewer}

\begin{itemize}
\item
  \emph{PLoS One}
\item
  \emph{Ecological Monographs}
\item
  \emph{Ecology}
\item
  \emph{Botany} (formerly \emph{The Canadian Journal of Botany})
\item
  \emph{Acta Oecologia}
\item
  \emph{Nature}
\end{itemize}

\subsection{Professional Memberships}\label{professional-memberships}

\begin{itemize}
\item
  \href{http://www.conbio.org/}{Socienty for Conservation Biology
  (SCB)}, 2008--present
\item
  \href{http://www.aaas.org/}{British Ecological Society (BES)}, 2008
\item
  \href{http://www.aaas.org/}{American Association for the Advancement
  of Science (AAAS)}, 2008
\item
  \href{http://www.esa.org/}{Ecological Society of America (ESA)},
  2005--present
\end{itemize}

\subsection{Meeting and Workshop
Organization}\label{meeting-and-workshop-organization}

\begin{itemize}
\item
  Workshop Coordinator, EU Sponsored White Paper Workshop on Foundation
  Species Genetics Research Directions, Flagstaff, AZ Spring 2011
\item
  Meeting Organizer, Western Mycorrhiza Gathering, Flagstaff, AZ, 2008
\item
  Workshop Organizer, IGERT Workshop: Bayesian Statistics in Ecology,
  Flagstaff, AZ, 2007
\item
  Meeting Organizer, Soil Ecology Society Conference, Moab, UT, 2007
\end{itemize}

\subsection{Education and Outreach}\label{education-and-outreach}

\begin{itemize}
\item
  Research featured in PBS documentary,
  \href{http://nau.edu/Research/Feature-Stories/A-Thousand-Invisible-Cords/}{One
  Thousand Invisible Cords}
\item
  Secretary, CO Plateau Chapter of the Society for Conservation Biology,
  2010--2012
\end{itemize}

\subsection{Student Services}\label{student-services}

\begin{itemize}
\item
  President, Biology Graduate Student Association (Northern Arizona
  University), 2009--2011
\item
  Email List Curator, Biology Graduate Student Association (Northern
  Arizona University), 2008
\item
  Social Coordinator, Biology Graduate Student Association (Northern
  Arizona University), 2006
\end{itemize}

\section{References}\label{references}

\begin{multicols}{2}

\begin{itemize}
\item
  Thomas G. Whitham\\ Regents' Professor\\ Northern Arizona
  University\\(928)
  523-7215\\\href{mailto:thomas.whitham@nau.edu}{\texttt{thomas.whitham@nau.edu}}
\item
  Stuart R. Borrett\\ Assistant Professor\\ University of North Carolina
  Wilmington\\(910)
  962-2411\\\href{mailto:borretts@uncw.edu}{borretts@uncw.edu}
\item
  Matthew A. Bowker\\ Assistant Professor\\ Northern Arizona
  University\\(928)
  523-9302\\\href{mailto:matthew.bowker@nau.edu}{matthew.bowker@nau.edu}
\end{itemize}

\begin{itemize}
\item
  Stephen M. Shuster\\ Professor of Zoology\\ Northern Arizona
  University\\(928)
  523-9302\\\href{mailto:stephen.shuster@nau.edu}{stephen.shuster@nau.edu}
\item
  Nancy C. Johnson\\ Professor of Biology\\ Northern Arizona
  University\\(928)
  523-6473\\\href{mailto:nancy.johnson@nau.edu}{nancy.johnson@nau.edu}
\item
  Aaron M. Ellison\\ Senior Research Fellow\\ Harvard Forest, Harvard
  University\\(978)
  724-3302\\\href{mailto:aellison@fas.harvard.edu}{aellison@fas.harvard.edu}
\end{itemize}

\end{multicols}

\end{document}
