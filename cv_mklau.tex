\documentclass[a4paper]{article}
\usepackage[margin=0.75in]{geometry}
\usepackage{multicol}
\usepackage[T1]{fontenc}
\usepackage{lmodern}
\usepackage{amssymb,amsmath}
\usepackage{ifxetex,ifluatex}
\usepackage{fixltx2e} % provides \textsubscript
% use upquote if available, for straight quotes in verbatim environments
\IfFileExists{upquote.sty}{\usepackage{upquote}}{}
\ifnum 0\ifxetex 1\fi\ifluatex 1\fi=0 % if pdftex
  \usepackage[utf8]{inputenc}
\else % if luatex or xelatex
  \ifxetex
    \usepackage{mathspec}
    \usepackage{xltxtra,xunicode}
  \else
    \usepackage{fontspec}
  \fi
  \defaultfontfeatures{Mapping=tex-text,Scale=MatchLowercase}
  \newcommand{\euro}{€}
\fi
% use microtype if available
\IfFileExists{microtype.sty}{\usepackage{microtype}}{}
\ifxetex
  \usepackage[setpagesize=false, % page size defined by xetex
              unicode=false, % unicode breaks when used with xetex
              xetex]{hyperref}
\else
  \usepackage[unicode=true]{hyperref}
\fi
\hypersetup{breaklinks=true,
            bookmarks=true,
            pdfauthor={},
            pdftitle={},
            colorlinks=true,
            citecolor=blue,
            urlcolor=blue,
            linkcolor=magenta,
            pdfborder={0 0 0}}
\urlstyle{same}  % don't use monospace font for urls
\setlength{\parindent}{0pt}
\setlength{\parskip}{6pt plus 2pt minus 1pt}
\setlength{\emergencystretch}{3em}  % prevent overfull lines
\setcounter{secnumdepth}{0}

\author{}
\date{}

\begin{document}

\textbf{\Large Matthew Kekoa Lau, PhD.} \\

\hline

\begin{multicols}{2}
\textit{Program Coordinator and Mahi‘ai} \\
MA‘O Organic Farms \\
\\
\textit{Lecturer} \\
Sustainable Community Food Systems Program \\
Department of Applied Sciences \\
University of Hawai‘i West O‘ahu \\

\columnbreak

Personal: \href{mailto:mk@mklau.info}{mk@mklau.info} \\
Farm: \href{mailto:matt@maoorganicfarms.org}{matt@maoorganicfarms.org} \\
University: \href{mailto:mklau3@hawaii.edu}{mklau3@hawaii.edu} \\
\\
LinkedIn: \href{www.linkedin.com/in/mklau-info/}{mklau-info} \\
ResearchGate: \href{www.researchgate.net/profile/Matthew_Lau2}{Matthew\_Lau2} \\

\end{multicols}

\section{Education}\label{education}

\begin{itemize}
\item Postdoctoral Fellow,
  \href{https://harvardforest.fas.harvard.edu/}{Harvard University, Harvard Forest}, 2014-2017
\item 
  Ph.D.~Biology, \href{http://www.nau.edu}{Northern Arizona University},
  \href{http://www.mpcer.nau.edu/igert/}{IGERT Fellow}, 2014

  \begin{itemize}
  \item
    \emph{Title}: The evolution of ecological networks.
  \item
    \emph{Awarded} Outstanding Dissertation (NAU Biological Sciences)
  \item
    \emph{Advisor}:
    \href{http://www6.nau.edu/biology/People/Faculty/Whitham/Whitham.htm}{Thomas
    G. Whitham}
  \end{itemize}
\item
  M.S. Biology, \href{http://www.nau.edu}{Northern Arizona University},
  2008
  \begin{itemize}
  \item
    \emph{Thesis}: Host species and site contribute to variation in
    foliar endophyte abundance, diversity and community composition.
  \item
    \emph{Advisor}:
    \href{http://www.nau.edu/~envsci/johnsonlab/index.htm}{Nancy C.
    Johnson}
  \end{itemize}
\item
  B.S. Biology, \href{http://www.humboldt.edu/~biosci/}{Humboldt State
  University}, 2004

  \begin{itemize}
  \item
    \emph{Emphasis}: Fungal Ecology
  \item
    \emph{Advisors}: Nathan J. Sanders and Terry W. Henkel
  \item
    \emph{Advanced Study}:
    \href{http://harvardforest.fas.harvard.edu/education/reu/reu.html}{Harvard
      Forest REU}, Summer 2003,
    \href{http://harvardforest.fas.harvard.edu/profiles/ellison.html}{Aaron
      M. Ellison}
  \end{itemize}
\end{itemize}

\section{Grants, Awards and Fellowships}\label{awards-and-fellowships}

\begin{itemize}
\item USDA NRCS Grant, Project Konohiki (MA‘O Organic Farms), 2022-2024
\item Young Scholar Grant (Chinese Academy of Sciences), 2021-2023
\item President's International Fellowship Initiative (Chinese Academy
  of Sciences), 2019--2021
\item External Research Fellow (Harvard University), 2018--2021
\item
  ARCS Foundation Scholarship (Phoenix Chapter), 2013--2014
\item
  Chateaubriand Fellowship (French Embassy), 2011--2012
\item
  NSF IGERT Fellowship (Northern Arizona University), 2008--2010
\item
  ARCS Foundation Scholarship (Phoenix Chapter), 2007--2008
\item
  ARCS Foundation Scholarship (Phoenix Chapter), 2006--2007
\item
  Northern Arizona University Minority Student Development Fellowship,
  2005--2008
\end{itemize}


\section{Teaching Experience}\label{teaching-experience}
\begin{itemize}
\item Climate Smart Agriculture Workshop Series, Farm Expansion Experience (FE‘E) Program, Summer 2022
\item Sustainable Community Food Systems in Hawai‘i (University of Hawai‘i West O‘ahu), Fall 2022
\item Agroecology (University of Hawai‘i West O‘ahu), Spring 2022
  Ecological Foodways (Lasell University), 2019-2021
\item
  Statistical Computing in R (Harvard Forest, Harvard
  University), Summer 2015-2018
\item
  Seminar on Ethics in Computational Ecology (Harvard Forest, Harvard
  University), Summer 2014
\item
  BIO326L Ecology Lab (Northern Arizona University), Fall 2012--Spring
  2013
\item
  \href{http://people.uncw.edu/borretts/courses/RworkshopUNCW.pdf}{Introduction
  to Programming in R (University of North Carolina Wilmington)}, Summer
  2011
\item
  BIO181 Introductory Biology: The Unity of Life (Northern Arizona
  University), Fall 2010
\item
  BIO680
  \href{http://www.mpcer.nau.edu/igert/eco_analysis_r.html}{Introduction
  to Ecological Analyses in R} (Northern Arizona University), Fall 2008
  and Fall 2009
\end{itemize}


\section{Research Positions}\label{research-experience}

\begin{itemize}
\item Research Fellow, Institute of Applied Ecology, Chinese Academy
  of Science, 2019-2021
\item External Research Fellow, Harvard University, 2018-2021
\item
  Postdoctoral Research Fellow, 
  \href{http://harvardforest.fas.harvard.edu/}{Harvard Forest}
  (\href{http://harvardforest.fas.harvard.edu/aaron-ellison}{Aaron M. Ellison}), Spring 2014-2018
\item
  Visiting Researcher,
  \href{https://www4.bordeaux-aquitaine.inra.fr/biogeco/}{Community
  Genetics Laboratory}
  (\href{http://www4.bordeaux-aquitaine.inra.fr/biogeco_eng/People/Former-members/Michalet-Richard}{Richard Michalet}), Fall 2011
\item
  Visiting Researcher, \href{http://people.uncw.edu/borretts/}{Systems
  Ecology and Ecoinformatics Lab}
  (\href{http://people.uncw.edu/borretts/people.html}{Stuart
  Borrett}), Summer 2011
\item
  Research Assistant, \href{http://www.poplar.nau.edu/}{Cottonwood
  Ecology Group}
  (\href{http://www.poplar.nau.edu/people.php?mode=showus\&user=tgw}{Dr.~Thomas
  G. Whitham}), 2010--2011
\item
  Research Assistant,
  \href{http://www.nau.edu/~envsci/johnsonlab/index.htm}{The Soil
  Ecology Lab}
  (\href{http://www.nau.edu/~envsci/johnsonlab/NCJ.htm}{Dr.~Nancy C.
  Johnson}), 2005--2008
\item
  Research Assistant, Humboldt State University Ant Ecology Lab
  \href{http://web.utk.edu/~nsanders/nate.html}{(Dr. Nathan J.
  Sanders)}, 2003
\end{itemize}

\section{Computer Software and Language
Proficiencies}\label{computer-software-and-language-proficiencies}

Computer: R, Python, Matlab, LaTeX, HTML, Bash, ruby, Emacs, git,
MacOS, Linux/Unix and Windows\\\\Human: English (native speaker),
French (not fluent), Mandarin (not fluent), Spanish (not fluent) and
Hawaiian (not fluent).




\section{Contributed Software}\label{contributed-software}

%% {Matthew K. Lau (2019) conetto: Generate network models from
%%   co-occurrence data.}

{Matthew K. Lau (2019) Rclean: The Rclean Package
  (v1.1.7). https://docs.ropensci.org/Rclean
  http://doi.org/10.5281/zenodo.3665732}

{Matthew K. Lau (2018) Rclean: A Tool for Writing Cleaner, more
  Transparent Code (v1.0.1). https://cran.r-project.org/package=Rclean
  http://doi.org/10.5281/zenodo.1208640}

{Thomas Pasquier, and Matthew K Lau. (2018). ProvTools/encapsulator:
  create a capsule for scientific projects
  (v0.1.1). http://doi.org/10.5281/zenodo.1199232}

{Matthew K. Lau, Nathan Justice and Aaron M. Ellison (2018) Virtual
  pitcher plant microecosystem simulator
  (v1.0.0). https://harvardforest.shinyapps.io/virtualpitcherplant/
  http://doi.org/10.5281/zenodo.1226372}

{Matthew K. Lau, Stuart R. Borrett, Pawandeep Singh and David E. Hines
  (2017) enaR: Tools for Ecological Network Analysis. R package
  version 3.0.0.}

{Matthew K. Lau and Raj Whitlock (2009) DiversitySampler: Functions
  for re-sampling a community matrix to compute diversity indices at
  different sampling levels.. R package version 2.0.}

{Matthew K. Lau (2009) DTK: Dunnett-Tukey-Kramer Pairwise Multiple
Comparison Test Adjusted for Unequal Variances and Unequal Sample Sizes.
R package version 3.0.}

\section{Publications}\label{publications}

\begin{itemize}
\item Lau~MK, Liu~B, Liang~Y (In Prep) The network structure of forest
  landscapes embodied in trade in China is highly regional and
  modular.
\item Lau~MK, Liang~Y (In Prep) Review of the application of network
  structural analyses to landscape extended multi-regional
  input-output models.
\item Lau~MK, Lamit~LJ, Borrett~SR, Bowker~MA, Naesbourg~R, Whitham~TG
  (In Review at \textit{Nature Ecology and Evolution}) Genotypic
  variation in a foundation tree species influences lichen interaction
  network structure.
\item Trisoviç~A, Pasquier~TFJ-M, Lau~MK, Seltzer~M, Crocas~M (2022)
  A large-scale study on quality and reproducibility of research
  outputs in R. Nature Scientific Data 9(1): 60. 
\item Liu~B, Wang~X, Lau~MK, and Liang~Y (In Review at \textit{Journal
  of Biogeography}) Divergent evolutionary tendency in functional
  traits of boreal forest understory vascular plant species.
\item Naesborg~R, Lau~MK, Michalet~R, Williams~C, Whitham~TG
  (Accepted) Tree genes affect rock lichen and plant understory
  communities: An example of trophic-independent
  interactions. Ecology.
\item Ma~T, Liang~Y, Sunde~M, Lau~MK, Liu~B, Wu~M, He~H (2021)
  Assessing the effects of climate variable and timescale selection on
  uncertainties in dryness/wetness trends in conterminous
  China. International Journal of Climatology.
\item Liu~B, Biswas~S, Yang~J, Liu~Z, He~H, Liang~Y, Lau~MK, Fang~Y,
  and Han~S (2020) Strong influences of stand age and topography on
  post-fire understory recovery in a Chinese boreal forest. Forest
  Ecology and Management. 473: 118307.
\item Lau~MK, Pasquier~FJ-M, Seltzer~M (2020). Rclean: A Tool for
  Writing Cleaner, More Transparent Code. Journal of Open Source
  Software, 5(46), 1312, https://doi.org/10.21105/joss.01312.
\item  Lau~MK, Ellison~AM, Nguyen~A, Penick~C, DeMarco~B, Gotelli~NJ,
  Sanders~NJ, Dunn~R and Helms~Cahan~S (2019) Draft Aphaenogaster
  genomes expand our view of ant genome size variation across climate
  gradients. PeerJ 7:e6447. 
\item  Lau~MK, Baiser~B, Northrop~A, Gotelli~NJ \& Ellison~AM (2018) Regime
  shifts and hysteresis in the pitcher-plant
  microecosystem. Ecological Modeling. 382: 1-8.
\item 
  Pasquier~T, Lau~MK, Han~X, Fong~E, Lerner~BS, Boose~ER, Crosas~M,
  Ellison~AM, Seltzer~M (2018) Sharing and Preserving Computational
  Analyses for Posterity with \texttt{encapsulator}. IEEE CiSE 20:
  111. \href{https://doi.org/10.1109/MCSE.2018.042781334}{https://doi.org/10.1109/MCSE.2018.042781334}
\item 
  Pasquier~T, Lau~MK, Trisovic~A, Boose~ER, Couturier~B, Crosas~M,
  Ellison~AM, Gibson~V, Jones~CR, Seltzer~M (2018) If these data could
  talk. Nature Sci. Data. 4: 170114.
\item  
  Ikeda, Dana H. and Max, Tamara L. and Allan, Gerard J. and Lau,
  Matthew K. and Shuster, Stephen M. and Whitham, Thomas G. (2017)
  Genetically informed ecological niche models improve climate change
  predictions. Glob. Chang. Biol. 23:164-176.
\item 
  Keith~AR, Bailey~JK, Lau~MK \& Whitham~TG (2016) Genetics-based
  interactions of foundation species affect community diversity,
  stability and network structure. Proc. R. Soc. B. 281.
\item  
  Lau~MK, Borrett~SR, Baiser~B, Gotelli~NJ \& Ellison~AM (2017)
  Ecological network metrics: opportunities for
  synthesis. Ecosphere. 8: e01900.
\item
  Lau~MK, Borrett~SR, Keith~AR, Shuster~SM \& Whitham~TG (2016)
  Genotypic variation in foundation species generates network
  structure that may drive community dynamics and
  evolution. Ecology. 97: 733-742.
\item
  Floate~KD, Godbout~J, Lau~MK, Whitham~TG, Isabel~N (2016)
  Plant-herbivore interactions in a trispecific hybrid swarm of
  cottonwoods: Genetic similarity and the hybrid bridge
  hypothesis. New Phytologist. 209: 832-844.
\item 
  Lamit~LJ, Busby~PE, Lau~MK, Compson~ZG, Wojtowicz~T, Keith~AR,
  Zinkgraf~MS, Schweitzer~JA, Shuster~SM, Gehring~CA, Whitham~TG
  (2015), Tree genotype mediates covariance among communities from
  microbes to lichens and arthropods. Journal of Ecology 103: 840–850.
\item
  Smith~DS, Lamit~LJ, Lau~MK, Gehring~CA \& Whitham~TG (2015) Change
  of plant traits by introduced elk negatively affects associated
  arthropod communities and network structure. Acta Oecologia 67: 8-16. 
\item 
  Smith~DS, Lau~MK, Jacobs~R, Monroy~JA, Shuster~SM, \& Witham~TG (2015)
  Introduced elk alter traits of a native plant and its plant-associated
  arthropod community. Oecologia DOI 10.1007/s00442-015-3362-y.
\item
  Borrett~SR \& Lau~MK (2014) enaR: An R package for Ecosystem Network
  Analysis. Methods in Ecology and Evolution 5: 1206-1213.
\item
  Lau~MK (2014) BOOK REVIEW: Grounding ecological networks. Ecology.
  95:2681--2682.
\item
  Flores-Rentería~L, Lau~MK, Lamit~LJ, \& Gehring CA (2014) An elusive
  ectomycorrhizal fungus reveals itself: A new species of Geopora
  (Pyronemataceae) associated with \textit{Pinus edulis}. Mycologia. DOI
  10.3852/13-263.
\item
  Lamit~LJ, Lau~MK, Sthultz~CM, Wooley~SC, Whitham~TG, \& Gehring~CA
  Tree genotype and genetically based growth traits structure twig
  endophyte communities. American Journal of Botany. DOI 10.3732/ajb.1400034.
\item
  Ikeda~DH, Bothwell~HM, Lau~MK, O'Neill~G, Grady~K, Ferrier~SM,
  Allan~G, Shuster~SM \& Whitham TG (2013) A genetics-based Universal
  Community Transfer Function for predicting the impacts of climate
  change on future communities. Functional Ecology 28:65--74.
\item
  Lau~MK, Arnold~EA \& Johnson~NC (2013)Factors influencing communities
  of foliar fungal endophytes in riparian woody plants. Fungal Ecology
  6: 365--378.
\item
  Álvarez-Sánchez~FJ, Johnson~NC, Antoninka~AJ, Chaudhary~VB, Lau~MK,
  Owen~SM, Sánchez-Gallen~I, Guadarrama~P, \& Castillo S (2012)
  Large-scale diversity patterns in spore communities of arbuscular
  mycorrhizal fungi. In M. Pagano, editor, \emph{Mycorrhiza: Occurrence
  in Natural and Restored Environments}, Nova Science Publishers, New
  York (USA).
\item
  Bowker~MA, Muñoz~A, Martinez~T \& Lau~MK 2012 Rare drought-induced
  mortality of juniper is enhanced by edaphic stressors and influenced
  by stand density. Journal of Arid Environments 76:9--16.
\item
  Lau~MK, Whitham~TG, Lamit~LJ \& Johnson~NC (2010) Ecological \&
  Evolutionary Interaction Network Exploration: Addressing the
  Complexity of Biological Interactions in Natural Systems with
  Community Genetics and Statistics. JIFS 7:17--25
\item
  Price~LB, Johnson~KE, Aziz~M, Lau~MK, Bowers~J, Ravel~J, Keim~PS,
  Serwadda~D, Wawer~MJ \& Gray~RH (2010) The effects of circumcision on
  the penis microbiome. PLoS One 5(1):e8422.
\item
  Chaudhary~VB, Lau~MK \& Johnson~NC (2008) Macroecology of microbes --
  biogeography of the Glomeromycota. In V. Ajit, editor,
  \emph{Mycorrhiza} (3rd Edition), Springer-Verlag, Germany.
\item
  Ellison~AM, Chen~J, Diaz~D, Kammerer-Burnham~C \& Lau~M (2005) Changes
  in ant community structure and composition associated with hemlock
  decline in New England. Pages 280-289 in B. Onkenand and R. Reardon,
  editors. \emph{Proceedings of the 3rd Symposium on Hemlock Woolly
  Adelgid in the Eastern United States}. U.S. Department of Agriculture
  -- U.S. Forest Service -- Forest Health Technology Enterprise Team,
  Morgantown, West Virginia.
\end{itemize}

\section{Presentations}\label{presentations}

\begin{itemize}
\item Lau~MK (2022) Project Konohiki: Collaborative Farmer Education
  in Hawai‘i. Sustainable Agriculture Education Society
  Meeting, University of Ohio. 
\item Lau~MK, Liu~B, \& Liang~Y (2020) Networks of forest trade in the
  Anthropocene. CSU San Bernadino.
\item Lau~MK (2020) Using R for open-science. University of
  Peshwar, Pakistan. 
\item Lau~MK (2019) Ecological Foodways: Landscape Opportunities at
  the Intersection of Food, Ecology, and Culture. ASLA, San Diego, CA
  (USA).
\item Lau~MK (2019) Genetic variation in a foundation tree generates
  ecological network structure. International Conference on Fire
  Disturbance Ecology, North East University, Cheng Chun (China).
\item Lau~MK (2019) Evolution of Ecological Networks: Climate Change
  and Human Landscapes. Institute of Applied Ecology, Chinese Academy
  of Sciences, Shenyang, China.
\item Lau~MK (2018) Automated code cleaning can help address the
  Reproducibility Crisis. Institute of Quantitative Social Sciences,
  Harvard University.
\item Lau~MK (2018) Ecological Network Evolution in the
  Anthropocene. Departmental Seminar, Institute of Applied Ecology,
  Chinese Academy of Sciences, Shenyang (China).
\item Lau~MK (2018) Ecological Network Evolution in the
  Anthropocene. Invited Speaker, North West Polytechnical University,
  Xian (China).
\item Lau~MK, Gotelli~NJ, Ellison~AM (2017) Provenance for
  reproducible multi-level foodwebs. Invited: Alfred Wagner Institute
  Symposium on Applied Foodweb Research. Sylt, Germany. 
\item Lau~MK \& Ellison~AM \emph{Temporal scales of coupled ecosystem
  processes provide a benchmark for alternate ecosystem states
  Photosynthesis and decomposition in a model micro-ecosystem.},
  Ecological Society of America Meeting (ESA), Baltimore, MD, August
  2015
\item Lau~MK \emph{Evolution and Ecological Networks: Merging
  Community Genetics and Network Ecology.} Ecological Society of
  America Meeting (ESA), Baltimore, MD, August 2015
\item
  Lau~MK, Borrett~SR \emph{enaR: Free, open-source tools for ecological
  network analysis.} Ecological Society of America Meeting (ESA),
  Minneapolis, MN, August 2013
\item
  Lau~MK, Lamit~LJ, Gehring~CA, and Whitham TG \emph{Cottonwood genetics
  influence lichen interaction network structure.} Université Bordeaux
  1, Talence, France, December 2011
\item
  Whitham~TG, Lau~MK, Lamit~LJ, Smith~DS, Busby~PE, Schweitzer~JA,
  Gehring~CA, Allan~GJ, Shuster~SM and Newcombe~G * A Community Genetics
  Approach for Understanding Microbial Community Structure and Feedbacks
  on a Foundation Tree Species.* Ecological Society of America Meeting
  (ESA), Pittsburgh, PA, August 2010
\item
  Lau~MK, Keith~AR and Whitham~TG \emph{Network structure is linked to
  the community stability of canopy arthropods associated with Populus
  angustifolia.} Ecological Society of America Meeting (ESA),
  Pittsburgh, PA, August 2010
\item
  Lau~MK, Johnson~NC, Whitham~TG, Hagenauer~LE, Lamit~LJ and Lonsdorf~EV
  \emph{A Community Genetics Approach for Understanding Complex
  Biological Interactions.} 7th International Symposium on Integrated
  Field Science, Tohoku University, Sendai, Japan, October 2009
\item
  Lau~MK, Hagenauer~LE and Whitham~TG \emph{Assemblage-structuring force
  of species interactions varies spatially and temporally: Co-occurrence
  analysis of canopy arthropod distributions.} Ecological Society of
  America Meeting (ESA), Albuquerque, NM, August 2009
\item
  Lau~MK, Johnson~NC \emph{Fungal foliar endophyte communities exhibit
  host species fidelity in woody plants of Arizona riparian forests.}
  Ecological Society of America Meeting (ESA), Milwaukee, WI, August
  2008
\item
  Lau~MK \emph{Unusual absence of asymptomatic fungal leaf endophytes of
  Populus fremontii: a potential phytochemical mechanism}. (poster)
  Ecological Society of America Meeting (ESA), San Jose, CA, August 2007
\item
  Whitewater~L, Lau~MK, Johnson~NC \emph{Investigating the potential for
  local adaptation of the arbuscular mycorrhizal fungus} . (poster) REU
  Summer Research Symposium, Northern Arizona University, Aug 2007.
\item
  Lau~MK, Johnson~NC \emph{Do AMF cultivate their favorite bacteria? A
  hypothesis for a potential mechanism of AMF adaptation}. (poster) 5th
  International Conference on Mycorrhiza (ICOM5), Granada, Spain, July
  2006

\end{itemize}

\section{Professional Activities}\label{professional-activities}

\subsection{Reviewer}\label{article-reviewer}

\begin{itemize}
\item
  \emph{Nature Ecology and Evolution}
\item
  \emph{Pedobiologia}
\item
  \emph{Ecological Modeling}
\item
  \emph{PNAS}
\item
  \emph{Ecology Letters}
\item
  \emph{PLoS One}
\item
  \emph{PLoS Computational Biology}
\item
  \emph{Ecological Monographs}
\item
  \emph{Ecology}
\item
  \emph{Journal of Ecology}
\item
  \emph{Botany} (formerly \emph{The Canadian Journal of Botany})
\item
  \emph{Acta Oecologia}
\item
  \emph{Nature}
\end{itemize}

\subsection{Professional Memberships}\label{professional-memberships}

\begin{itemize}
\item
  \href{http://www.conbio.org/}{Society for Conservation Biology
  (SCB)}
\item
  \href{http://www.aaas.org/}{British Ecological Society (BES)}, 2008
\item
  \href{http://www.aaas.org/}{American Association for the Advancement
  of Science (AAAS)}
\item
  \href{http://www.esa.org/}{Ecological Society of America (ESA)}
\end{itemize}


\subsection{Professional Service}\label{professional-service}

\begin{itemize}
\item 
  Workshop Organizer, Ecological Foodways: Landscape Opportunities at
  the Intersection of Food, Ecology, and Culture (ASLA), San Diego,
  CA, 2019.
\item 
  Workshop Organizer, Introduction to Ecological Network Analysis,
  Ecological Society of America Meeting (ESA), Baltimore, MD, 2015.
\item
  Workshop Coordinator, EU Sponsored White Paper Workshop on Foundation
  Species Genetics Research Directions, Flagstaff, AZ Spring 2011
\item
  Meeting Organizer, Western Mycorrhiza Gathering, Flagstaff, AZ, 2008
\item
  Workshop Organizer, IGERT Workshop: Bayesian Statistics in Ecology,
  Flagstaff, AZ, 2007
\item
  Meeting Organizer, Soil Ecology Society Conference, Moab, UT, 2007
\end{itemize}


\end{document}
